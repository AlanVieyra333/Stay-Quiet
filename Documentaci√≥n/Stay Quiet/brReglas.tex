\section{Reglas de Negocio}

\begin{BussinesRule}{RN-01}{Campos nulos}
	\BRitem[Descripción:] Ningún dato en el formulario puede ser nulo.
	\BRitem[Nivel:] Obligatorio.
\end{BussinesRule}

\begin{BussinesRule}{RN-02}{Formato general de registro}
	\BRitem[Nombre:] Formato del nombre.
	\BRitem[Descripción:] El nombre esta compuesto por:
		\begin{itemize}
			\item Nombre Completo.
		\end{itemize}
		Todo el nombre debe estar compuesto por letras.
		Ejemplo 
		Nombre: Luis Ángel Martínez Gómez.
	\BRitem[Nivel:] Obligatorio.\\\\

	\BRitem[Teléfono:] Formato de nombre de usuario.
	\BRitem[Descripción:] Debe de temer números y letras mayúsculas y/o minúsculas.
		Ejemplo
		Nombre: Abiran
		Correo: abisaher@gmail.com
		NombreUsuario: Abiran55
	\BRitem[Nivel:] Obligatorio.\\\\

	\BRitem[Teléfono:] Formato del teléfono.
	\BRitem[Descripción:] El teléfono debe estar formado solamente con números.
		Ejemplo
		Teléfono: 58601859 o 5572753650
	\BRitem[Nivel:] Obligatorio.\\\\

	\BRitem[Teléfono:] Formato del correo.
	\BRitem[Descripción:] El correo tiene el siguiente formato: example@mail.com.
		Ejemplo
		Correo: example@mail.com
	\BRitem[Nivel:] Obligatorio.\\\\

	\BRitem[Contraseña:] Formato de la contraseña.
	\BRitem[Descripción:] La contraseña debe tener un tamaño mínimo de 8 caracteres, y un máximo de 16 caracteres, la cual está compuesta por:
		\begin{itemize}
			\item Letras mayúsculas.
			\item Letras minúsculas.
			\item Dígitos.
			\item Caracteres especiales.
		\end{itemize}
	\BRitem[Nivel:] Obligatorio.\\\\

	\BRitem[Repetir contraseña:] Introduce la misma contraseña que ingresó en el paso anterior.
	\BRitem[Descripción:] La contraseña escrita deberá ser la misma para confirmar que es la correcta.
	\BRitem[Nivel:] Obligatorio.\\\\
\end{BussinesRule}

\begin{BussinesRule}{RN-03}{Usuario registrado}
	\BRitem[Descripción:] Verificar que el usuario este dado de alta en la aplicación.
	\BRitem[Nivel:] Obligatorio.
\end{BussinesRule}

\begin{BussinesRule}{RN-04}{Checkbox}
	\BRitem[Descripción:] Verificar que el usuario quiera guardar sus credenciales en el login y entrar automáticamente a la aplicación.
	\BRitem[Nivel:] Opcional
\end{BussinesRule}

\begin{BussinesRule}{RN-05}{Nombre de usuario único}
	\BRitem[Descripción:] El nombre de usuario sera único para cada uno de los usuarios de la aplicación móvil. Por lo tanto no existirá otro usuario con el mismo nombre de usuario.
	\BRitem[Nivel:] Obligatorio
\end{BussinesRule}

\begin{BussinesRule}{RN-06}{Peticion de ubicacion}
	\BRitem[Descripción:] El usuario a quien se le pedirá su ubicación debe dar su consentimiento para dar su ubicación.
	\BRitem[Nivel:] Obligatorio
\end{BussinesRule}

\begin{BussinesRule}{RN-07}{Parámetros de rango}
	\BRitem[Descripción:] El usuario podrá marcar un rango mínimo de 20 metros y un máximo de 1 kilómetro.
	\BRitem[Nivel:] Obligatorio
\end{BussinesRule}

\begin{BussinesRule}{RN-08}{Contacto fuera del rango}
	\BRitem[Descripción:] Se le notificara al usuario cuando el contacto este fuera del rango marcado. Esto mediante de alertas donde se parara hasta que el usuario lo desactive.
	\BRitem[Nivel:] Obligatorio
\end{BussinesRule}

