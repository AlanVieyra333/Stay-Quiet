\begin{UseCase}{CU-05}{Cambiar contraseña}{
	Se describe el comportamiento y funcionamiento de la aplicación para la modificación de la contraseña del usuario.
}
		\UCitem{Actor}{Usuario}
		\UCitem{Propósito}{
		Poder cambiar la contraseña del usuario
		}
		\UCitem{Entradas}{
			Seleccionar el campo a modificar:
			\begin{itemize}
				\item Contraseña.
			\end{itemize}
		 }
		\UCitem{Salidas}{
			\begin{itemize}
				\item \MSGref{MSJ-01}{Campos Incompletos}
				\item \MSGref{MSJ-04}{Formato de contraseña incorrecto}
				\item \MSGref{MSJ-06}{Error en conexión}
				\item \MSGref{MSJ-09}{Contraseña incorrecta}
			\end{itemize}
		}
		\UCitem{Precondiciones}{El usuario debe estar registrado en el sistema.
		}
		\UCitem{Postcondiciones}{
			El usuario podrá modificar su contraseña la veces que sean necesarias.
		}
		\UCitem{Autor}{
			\begin{itemize}
				\item Salas Hernández Abiran Natanael
			\end{itemize}
		}
	\end{UseCase}
	%-------------------------------------- COMIENZA descripción Trayectoria Crear
	\begin{UCtrayectoria}{Principal}
		\UCpaso Reproducir el \UCref{CU-01}{Iniciar Sesión}
		\UCpaso[\UCactor] Pulsa en el menú superior derecho.
		\UCpaso[\UCsist] Despliega el menú con tres opciones \IUbutton{Perfil},\IUbutton{Seguridad} y \IUbutton{Cerrar Sesión}.
		\UCpaso[\UCactor] Pulsa en el  \IUbutton{Seguridad}.
		\UCpaso[\UCsist] Despliega la pantalla \IUref{IU-08}{Nueva contraseña}.
		\UCpaso[\UCactor] Captura los datos necesarios.
		\UCpaso[\UCactor] Pulsa en el  \IUbutton{CONTINUAR}.
		\UCpaso[\UCsist] Verifica la regla de negocio \BRref{RN-01}{Campos nulos}. \Trayref{A}
		\UCpaso[\UCsist] Verifica la regla de negocio \BRref{RN-02}{Formato general de registro} para el formato de contraseña. \Trayref{B}
		\UCpaso[\UCsist] Entra a la base de datos con los datos ingresados. \Trayref{C} \Trayref{D} \Trayref{E}
		\UCpaso[\UCsist] Muestra el mensaje \MSGref{MSJ-12}{Cambio contraseña}.		
		\UCpaso[\UCsist] Despliega la pantalla \IUref{IU-03}{Home}.
	\end{UCtrayectoria}
	%-------------------------------------Trayectoras alternativas
	\begin{UCtrayectoriaA}{A}{Condición: Faltan datos en el formulario.}
		\UCpaso[\UCsist] Muestra mensaje \MSGref{MSJ-01}{Campos Incompletos}
		\UCpaso[\UCsist] Continúa trayectoria en el paso anterior de la trayectoria  principal que invocó esta trayectoria alternativa. \Trayref{Principal}
	\end{UCtrayectoriaA}
	%-------------------------------------Trayectoras alternativas
	\begin{UCtrayectoriaA}{B}{Condición: El actor captura la información de manera incorrecta.}
		\UCpaso[\UCsist] Dependiendo del error sera el mensaje:
			\begin{itemize}
				\item \MSGref{MSJ-04}{Formato de contraseña incorrecto}
			\end{itemize}
		\UCpaso[\UCsist] Continúa trayectoria en el paso anterior de la trayectoria principal que invocó esta trayectoria alternativa. \Trayref{Principal}
	\end{UCtrayectoriaA}
	%-------------------------------------Trayectoras alternativas
	\begin{UCtrayectoriaA}{C}{Condición: Falló la conexión en la base de datos}
		\UCpaso[\UCsist] El sistema mostrará el mensaje \MSGref{MSJ-06}{Error en conexión}
		\UCpaso[\UCsist] Regresa al paso 5 de la trayectoria principal. \Trayref{Principal}
	\end{UCtrayectoriaA}
	%-------------------------------------Trayectoras alternativas
	\begin{UCtrayectoriaA}{D}{Condición: Contraseña incorrecta}
		\UCpaso[\UCsist] El sistema mostrará el mensaje \MSGref{MSJ-09}{Contraseña incorrecta}
		\UCpaso[\UCsist] Continúa trayectoria en el paso anterior de la trayectoria  principal que invocó esta trayectoria alternativa. \Trayref{Principal}
	\end{UCtrayectoriaA}
