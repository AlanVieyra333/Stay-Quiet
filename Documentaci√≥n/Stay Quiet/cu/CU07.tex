\begin{UseCase}{CU-07}{Información}{
	Se describe el comportamiento y funcionamiento de la aplicación para visualizar la información de un contacto.
}
		\UCitem{Actor}{Usuario}
		\UCitem{Propósito}{Ver la información de un contacto}
		\UCitem{Entradas}{
			Se ingresa la información:
			\begin{itemize}
				\item Ninguno
			\end{itemize}
		 }
		\UCitem{Salidas}{
			\begin{itemize}
				\item Ninguno
			\end{itemize}
		}
		\UCitem{Precondiciones}{
			El usuario deberá haber agregado a un contacto.
		}
		\UCitem{Postcondiciones}{
			El usuario podrá consultar la información del contacto.
		}
		\UCitem{Autor}{
			\begin{itemize}
				\item Salas Hernández Abiran Natanael
			\end{itemize}
		}
	\end{UseCase}
	%-------------------------------------- COMIENZA descripción Trayectoria Crear
	\begin{UCtrayectoria}{Principal}
		\UCpaso Reproducir el \UCref{CU-01}{Iniciar Sesión}
		\UCpaso[\UCactor] Selecciona de la lista, al contacto deseado y presiona el botón \IUbutton{Información}. \Trayref{A} \Trayref{B}
		\UCpaso[\UCsist] Despliega una carpeta hacia abajo con la información del contacto.
	\end{UCtrayectoria}
	%-------------------------------------Trayectoras alternativas
	\begin{UCtrayectoriaA}{A}{Condición: Falló la conexión en la base de datos}
		\UCpaso[\UCsist] El sistema mostrará el mensaje \MSGref{MSJ-06}{Error en conexión}
		\UCpaso[\UCsist] Regresa al paso 4 de la trayectoria principal. \Trayref{Principal}
	\end{UCtrayectoriaA}
	%-------------------------------------Trayectoras alternativas
	\begin{UCtrayectoriaA}{B}{Condición: No existen contactos}
		\UCpaso[\UCsist] Reproducir el \UCref{CU-06}{Agregar}. \Trayref{Principal}
	\end{UCtrayectoriaA}