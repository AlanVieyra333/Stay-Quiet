\begin{UseCase}{CU1.2}{Registrar usuario}{
	Se describe el comportamiento y funcionamiento de la aplicación para el registro del usuario.
}
		\UCitem{Versión}{1.0}
		\UCitem{Actor}{Protector.}
		\UCitem{Propósito}{Tener un control de los usuarios nuevos para así poder entrar a su perfil correspondiente y pueda hacer uso de la funciones de la aplicación.}
		\UCitem{Entradas}{
			Se ingresa la información:
			\begin{itemize}	
				\item Nombre.
				\item Apellidos.
				\item Correo Electrónico.
				\item Numero de Celular.
				\item Contraseña y/o PIN.
			\end{itemize}
		 }
		\UCitem{Salidas}{
			\begin{itemize}
				\item \MSGref{MSJ1.1}{Campos Incompletos}
				\item \MSGref{MSJ1.2}{Formato de correo incorrecto}
				\item \MSGref{MSJ1.3}{Error en conexión}
				\item \MSGref{MSJ1.5}{Datos ya utilizados}
			\end{itemize}
		}
		\UCitem{Precondiciones}{La aplicación debe estar instalada en el celular.
		}
		\UCitem{Postcondiciones}{
			El usuario podrá hacer uso de la aplicación según sea su perfil.
		}
		\UCitem{Autor}{
			\begin{itemize}
				\item Salas Hernandez Abiran
			\end{itemize}
		}
		\UCitem{Estatus}{Revisión}
	\end{UseCase}
	%-------------------------------------- COMIENZA descripción Trayectoria Crear
	\begin{UCtrayectoria}{Principal}
		\UCpaso[\UCactor] Pulsar en el icono de la aplicación.
		\UCpaso[\UCsist] Despliega la pantalla \IUref{IU1.1}{Pantalla Principal}.
		\UCpaso[\UCactor] Pulsar en el botón \IUbutton{REGISTRATE!} de la pantalla.
		\UCpaso[\UCsist] Despliega la pantalla \IUref{IU1.2}{Registrar Usuario}.
		\UCpaso[\UCactor] Captura los datos necesarios en la misma pantalla.
		\UCpaso[\UCactor] Pulsar el botón \IUbutton{REGISTRATE} de la pantalla.
		\UCpaso[\UCsist] Verifica la regla de negocio \BRref{RN1.1}{Campos nulos}. \Trayref{A}
		\UCpaso[\UCsist] Verifica la regla de negocio \BRref{RN1.2}{Formato de Correo Electrónico}. \Trayref{B}
		\UCpaso[\UCsist] Entra a la base de datos con los datos ingresados. \Trayref{C}
		\UCpaso[\UCsist] Verifica la regla de negocio \BRref{RN1.3}{Usuario registrado en el sistema}. \Trayref{D}
		\UCpaso[\UCsist] Muestra la pantalla \IUref{IU1.3}{Pantalla Inicial}.
	\end{UCtrayectoria}
	%-------------------------------------Trayectoras alternativas
	\begin{UCtrayectoriaA}{A}{Condición: Faltan datos en el formulario.}
		\UCpaso[\UCsist] Muestra mensaje \MSGref{MSJ1.1}{Campos Incompletos}
		\UCpaso[\UCsist] Continúa trayectoria en el paso  4 de la trayectoria  principal que invocó esta trayectoria alternativa. \Trayref{Principal}
	\end{UCtrayectoriaA}
	%-------------------------------------Trayectoras alternativas
	\begin{UCtrayectoriaA}{B}{Condición: El actor ingresa formato de correo incorrecto.}
		\UCpaso[\UCsist] Muestra mensaje \MSGref{MSJ1.2}{Formato de correo incorrecto}
		\UCpaso[\UCsist] Continúa trayectoria en el paso 4 de la trayectoria  principal que invocó esta trayectoria alternativa. \Trayref{Principal} 
	\end{UCtrayectoriaA}
	%-------------------------------------Trayectoras alternativas
	\begin{UCtrayectoriaA}{C}{Condición: Falló la conexión en la base de datos}
		\UCpaso[\UCsist] El sistema mostrará el mensaje \MSGref{MSJ1.3}{Error en conexión}
		\UCpaso[\UCsist] Regresa al paso 4 de la trayectoria principal. \Trayref{Principal}
	\end{UCtrayectoriaA}	
	%-------------------------------------Trayectoras alternativas
		\begin{UCtrayectoriaA}{D}{Condición: Usuario ya registrado y/o correo y/o teléfono ya utilizados}
			\UCpaso[\UCsist] El sistema mostrará el mensaje \MSGref{MSJ1.5}{Datos ya utilizados}
			\UCpaso[\UCsist] Continúa trayectoria en el paso 4 de la trayectoria  principal que invocó esta trayectoria alternativa. \Trayref{Principal}
	\end{UCtrayectoriaA}
	%-------------------------------------Trayectoras alternativas