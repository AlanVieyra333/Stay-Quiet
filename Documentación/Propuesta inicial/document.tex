      \documentclass[10pt]{article}  

%%%%%%%% PREÁMBULO %%%%%%%%%%%%
\title{Plantilla para prácticas de ESCOM}
\usepackage[spanish]{babel} %Indica que escribiermos en español
\usepackage[utf8]{inputenc} %Indica qué codificación se está usando ISO-8859-1(latin1)  o utf8  
\usepackage{amsmath} % Comandos extras para matemáticas (cajas para ecuaciones,
% etc)
\usepackage{amssymb} % Simbolos matematicos (por lo tanto)
\usepackage{graphicx} % Incluir imágenes en LaTeX
\usepackage{color} % Para colorear texto
\usepackage{subfigure} % subfiguras
\usepackage{float} %Podemos usar el especificador [H] en las figuras para que se
% queden donde queramos
\usepackage{capt-of} % Permite usar etiquetas fuera de elementos flotantes
% (etiquetas de figuras)
\usepackage{sidecap} % Para poner el texto de las imágenes al lado
    \sidecaptionvpos{figure}{c} % Para que el texto se alinie al centro vertical
\usepackage{caption} % Para poder quitar numeracion de figuras
\usepackage{commath} % funcionalidades extras para diferenciales, integrales,
% etc (\od, \dif, etc)
\usepackage{cancel} % para cancelar expresiones (\cancelto{0}{x})
 
\usepackage{anysize}                    % Para personalizar el ancho de  los márgenes
\marginsize{2cm}{2cm}{2cm}{2cm} % Izquierda, derecha, arriba, abajo

\usepackage{appendix}
\renewcommand{\appendixname}{Apéndices}
\renewcommand{\appendixtocname}{Apéndices}
\renewcommand{\appendixpagename}{Apéndices} 

% Para que las referencias sean hipervínculos a las figuras o ecuaciones y
% aparezcan en color
\usepackage[colorlinks=true,plainpages=true,citecolor=blue,linkcolor=blue]{hyperref}
%\usepackage{hyperref} 
% Para agregar encabezado y pie de página
\usepackage{fancyhdr} 
\pagestyle{fancy}
\fancyhf{}
\fancyhead[L]{\footnotesize ESCOM} %encabezado izquierda
\fancyhead[R]{\footnotesize IPN}   % dereecha
\fancyfoot[R]{\footnotesize Propuesta Inicial}  % Pie derecha
\fancyfoot[C]{\thepage}  % centro
\fancyfoot[L]{\footnotesize S.E For Mobile Devices}  %izquierda
\renewcommand{\footrulewidth}{0.4pt}


\usepackage{listings} % Para usar código fuente
\definecolor{dkgreen}{rgb}{0,0.6,0} % Definimos colores para usar en el código
\definecolor{gray}{rgb}{0.5,0.5,0.5} 
% configuración para el lenguaje que queramos utilizar
\lstset{language=Matlab,
   keywords={break,case,catch,continue,else,elseif,end,for,function,
      global,if,otherwise,persistent,return,switch,try,while},
   basicstyle=\ttfamily,
   keywordstyle=\color{blue},
   commentstyle=\color{red},
   stringstyle=\color{dkgreen},
   numbers=left,
   numberstyle=\tiny\color{gray},
   stepnumber=1,
   numbersep=10pt,
   backgroundcolor=\color{white},
   tabsize=4,
   showspaces=false,
   showstringspaces=false}

\newcommand{\sen}{\operatorname{\sen}}  % Definimos el comando \sen para el seno
%en español

\title{Plantilla para trabajo terminal I de UPIITA}

%%%%%%%% TERMINA PREÁMBULO %%%%%%%%%%%%

\begin{document}

%%%%%%%%%%%%%%%%%%%%%%%%%%%%%%%%%% PORTADA %%%%%%%%%%%%%%%%%%%%%%%%%%%%%%%%%%%%%%%%%%%%
                                                                                    %%%
\begin{center}                                                                      %%%
\newcommand{\HRule}{\rule{\linewidth}{0.5mm}}                                   %%%\left
                                                                                    %%%
\begin{minipage}{0.48\textwidth} \begin{flushleft}
\includegraphics[scale = 0.1]{Imagen/ESCOM-LOGO}
\end{flushleft}\end{minipage}
\begin{minipage}{0.48\textwidth} \begin{flushright}
\includegraphics[scale = 0.35]{Imagen/IPN-LOGO}
\end{flushright}\end{minipage}

                                                                                    %%%
\vspace*{-1.5cm}                                %%%
                                                                                    %%% 
\textsc{\huge Instituto Polit\'ecnico\\ \vspace{5px} Nacional}\\[1.5cm] 

\textsc{\LARGE Escuela superior de c\'omputo}\\[1.5cm]                                                   %%%

\begin{minipage}{0.9\textwidth} 
\begin{center}                                                                                  %%%
\textsc{\LARGE Propuesta Inicial}
\end{center}
\end{minipage}\\[0.5cm]
%%%
                                                                                    %%%
            \vspace*{1cm}                                                                       %%%
                                                                                    %%%
\HRule \\[0.4cm]                                                                    %%%
{ \huge \bfseries APP - State Quiet}\\[0.4cm]  %%%
                                                                                    %%%
\HRule \\[1.5cm]                                                                    %%%
                                                                                %%%
                                                                                    %%%
\begin{minipage}{0.46\textwidth}                                                    %%%
\begin{flushleft} \large                                                            %%%
\emph{Autores:}\\ 
Hernandez Soriano Gerardo  \\
Moreno Sánchez José Rodolfo\\
Perez Montiel Ulises\\
Salas Hernandez Abiran Natanael\\ 
Rincón Vieyra Alan 
%%%
            %\vspace*{2cm}  
                                                                %%%
                                                                %%%
\end{flushleft}                                                                     %%%
\end{minipage}      
                                                                %%%
\begin{minipage}{0.52\textwidth}        
\vspace{-0.6cm}                                         %%%
\begin{flushright} \large                                                           %%%
\emph{Asesor:} \\                                                                 %%%
Ulises Vélez Saldaña 1\\                                             %%%
\end{flushright}                                                                    %%%
\end{minipage}  
\vspace*{1cm}
%\begin{flushleft}
    
%\end{flushleft}
%%%
        \flushleft{\textbf{\Large S.E For Mobile Devices} }\\                                                                     %%%
\vspace{2cm}                                                                                
\begin{center}                                                                                  
{\large \today}                                                                 %%%
            \end{center}                                                                        
\end{center}                                                                        
                                                                                    
\newpage                                                                        
%%%%%%%%%%%%%%%%%%%% TERMINA PORTADA %%%%%%%%%%%%%%%%%%%%%%%%%%%%%%%%

\tableofcontents 

\newpage



\section{Contexto.}
Para reportar una persona desaparecida o extraviada se requiere esperar 72 horas. \\

Tiempo durante el cual una familia o un conjunto de personas designadas al cuidado de alguien requieren valerse de sus recursos para localizar a la persona de quien no saben donde se encuentra. \\

La señal de GPS cubre grandes áreas donde gran numero de personas habitan, por lo cual es altamente probable que se extravié un ser humano dentro de esta área   
 


\section{Problema.}
Hay seres humanos propensos a extraviarse o sufrir secuestro, por ejemplo, niños, ancianos o personas que padecen de sus facultades mentales

\section{Solución propuesta.}
Desarrollar una app móvil que mediante el uso de GPS apoye a supervisar la ubicación de una persona dentro de un área pre establecida  

\section{Requerimientos.}
\subsection{Uso de GPS.} 
La aplicación utiliza el servicio de GPS del celular para establecer la ubicación en un mapa  
\subsection{Dos Perfiles.}
- El perfil de usuario, permite tener contactos con los cuales puede compartir su ubicación cada determinado tiempo, y se llama observador cuando se vincula a un perfil protegido\\

El perfil de protegido, pensado para las personas que se quiere cuidar, la aplicación se mantiene funcionando como GPS, se requiere introducir un password para acceder a la aplicación 

\subsection{Selección de área de cuidado .}

El observador puede establecer una zona geografica entorno a un punto fijo en el mapa ó a una distancia de una trayectoria 

\subsection{Mensajes de alerta.}

Si el dispositivo con el perfil protegido sale del área de cuidado el observador recibe una alerta y puede iniciar una llamada 

Si el dispositivo con el perfil protegido tiene batería baja, pocos datos o señal débil, se envía un mensaje al observador


\end{document}