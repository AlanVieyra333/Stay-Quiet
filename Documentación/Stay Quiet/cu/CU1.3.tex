\begin{UseCase}{CU1.3}{Editar Protector}{
	Se describe el comportamiento y funcionamiento de la aplicación para la modificación de los datos del Protector.
}
		\UCitem{Versión}{1.0}
		\UCitem{Actor}{Protector.}
		\UCitem{Propósito}{Poder actualizar, modificar los datos del usuario}
		\UCitem{Entradas}{
			Seleccionar el campo a modificar:
			\begin{itemize}
				\item Nombre.
				\item Correo Electrónico.
				\item Numero de Celular.
			\end{itemize}
		 }
		\UCitem{Salidas}{
			\begin{itemize}
				\item \MSGref{MSJ1.1}{Campos Incompletos}
				\item \MSGref{MSJ1.2}{Formato de nombre incorrecto}
				\item \MSGref{MSJ1.3}{Formato de teléfono incorrecto}
				\item \MSGref{MSJ1.6}{Error en conexión}
				\item \MSGref{MSJ1.8}{Datos utilizados}
				\item \MSGref{MSJ1.9}{Contraseña incorrecta}
				\item \MSGref{MSJ1.10}{Formato de correo incorrecto}
			\end{itemize}
		}
		\UCitem{Precondiciones}{El usuario debe estar registrado en el sistema.
		}
		\UCitem{Postcondiciones}{
			El usuario podrá modificar sus datos la veces que sean necesarias.
		}
		\UCitem{Autor}{
			\begin{itemize}
				\item Salas Hernández Abiran Natanael
			\end{itemize}
		}
		\UCitem{Estatus}{Revisión}
	\end{UseCase}
	%-------------------------------------- COMIENZA descripción Trayectoria Crear
	\begin{UCtrayectoria}{Principal}
		\UCpaso[\UCactor] Pulsa en el icono de la aplicación.
		\UCpaso[\UCsist] Despliega la pantalla \IUref{IU1.1}{Pantalla Principal}.
		\UCpaso[\UCactor] Captura los datos necesarios en la misma pantalla.
		\UCpaso[\UCactor] Pulsa en el checkbox para recordar los datos.
		\UCpaso[\UCactor] Pulsa en el botón \IUbutton{ENTRAR} de la pantalla.
		\UCpaso[\UCsist] Verifica la regla de negocio \BRref{RN1.1}{Campos nulos}. \Trayref{A}
		\UCpaso[\UCsist] Verifica la regla de negocio \BRref{RN1.2}{Formato general para el registro de un Protector}  para el formato del correo. \Trayref{B}
		\UCpaso[\UCsist] Verifica la regla de negocio (Checkbox). \Trayref{C}
		\UCpaso[\UCsist] Entra a la base de datos con los datos ingresados. \Trayref{D}
		\UCpaso[\UCsist] Verifica la regla de negocio \BRref{RN1.3}{Usuario registrado en el sistema}. \Trayref{E}
		\UCpaso[\UCsist] Muestra la pantalla \IUref{IU1.3}{Pantalla Inicial}.
		\UCpaso[\UCactor] Pulsa en el menú superior derecho.
		\UCpaso[\UCsist] Despliega el menú con tres opciones \IUbutton{Perfil},\IUbutton{Seguridad} y \IUbutton{Cerrar Sesión}.
		\UCpaso[\UCactor] Pulsa en el  \IUbutton{Perfil}.
		\UCpaso[\UCsist] Despliega la pantalla \IUref{IU1.4}{Perfil Usuario}.
		\UCpaso[\UCactor] Selecciona el campo a modificar.
		\UCpaso[\UCactor] Pulsa en el  \IUbutton{GUARDAR}.
		\UCpaso[\UCsist] Verifica la regla de negocio \BRref{RN1.1}{Campos nulos}. \Trayref{A}
		\UCpaso[\UCsist] Verifica la regla de negocio \BRref{RN1.2}{RN1.2}{Formato general para el registro de un Protector}. \Trayref{B}
		\UCpaso[\UCsist] Despliega la pantalla \IUref{IU1.5}{Ingresa Contraseña}.
		\UCpaso[\UCactor] Captura su contraseña en la pantalla.
		\UCpaso[\UCactor] Pulsa en el  \IUbutton{CONTINUAR}.
		\UCpaso[\UCsist] Entra a la base de datos con los datos ingresados. \Trayref{C} \Trayref{D} \Trayref{E}
		\UCpaso[\UCsist] Despliega la pantalla \IUref{IU1.3}{Pantalla Inicial}.
	\end{UCtrayectoria}
	%-------------------------------------Trayectoras alternativas
	\begin{UCtrayectoriaA}{A}{Condición: Faltan datos en el formulario.}
		\UCpaso[\UCsist] Muestra mensaje \MSGref{MSJ1.1}{Campos Incompletos}
		\UCpaso[\UCsist] Continúa trayectoria en el paso anterior de la trayectoria  principal que invocó esta trayectoria alternativa. \Trayref{Principal}
	\end{UCtrayectoriaA}
	%-------------------------------------Trayectoras alternativas
	\begin{UCtrayectoriaA}{B}{Condición: El actor captura la informacion de manera incorrecta.}
		\UCpaso[\UCsist] Dependiendo del error sera el mensaje:
			\begin{itemize}
			\item \MSGref{MSJ1.2}{Formato de nombre incorrecto}
			\item \MSGref{MSJ1.3}{Formato de teléfono incorrecto}
			\item \MSGref{MSJ1.4}{Formato de contraseña incorrecto}
			\item \MSGref{MSJ1.5}{Contraseña no coincide}
			\item \MSGref{MSJ1.10}{Formato de correo incorrecto}
			\end{itemize}
		\UCpaso[\UCsist] Continúa trayectoria en el paso anterior de la trayectoria principal que invocó esta trayectoria alternativa. \Trayref{Principal}
	\end{UCtrayectoriaA}
	%-------------------------------------Trayectoras alternativas
	\begin{UCtrayectoriaA}{C}{Condición: Falló la conexión en la base de datos}
		\UCpaso[\UCsist] El sistema mostrará el mensaje \MSGref{MSJ1.6}{Error en conexión}
		\UCpaso[\UCsist] Regresa al paso 1 de la trayectoria principal. \Trayref{Principal}
	\end{UCtrayectoriaA}
	%-------------------------------------Trayectoras alternativas
	\begin{UCtrayectoriaA}{D}{Condición: Contraseña incorrecta}
		\UCpaso[\UCsist] El sistema mostrará el mensaje \MSGref{MSJ1.9}{Contraseña incorrecta}
		\UCpaso[\UCsist] Continúa trayectoria en el paso anterior de la trayectoria  principal que invocó esta trayectoria alternativa. \Trayref{Principal}
	\end{UCtrayectoriaA}
	%-------------------------------------Trayectoras alternativas
	\begin{UCtrayectoriaA}{E}{Condición: Usuario ya registrado y/o correo y/o teléfono ya utilizados}
		\UCpaso[\UCsist] El sistema mostrará el mensaje \MSGref{MSJ1.8}{Datos ya utilizados}
		\UCpaso[\UCsist] Continúa trayectoria en el paso 4 de la trayectoria  principal que invocó esta trayectoria alternativa. \Trayref{Principal}
	\end{UCtrayectoriaA}
	%-------------------------------------Trayectoras alternativas
