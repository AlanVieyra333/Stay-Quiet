\begin{UseCase}{CU1.5}{Editar Protegido}{
	Se describe el comportamiento y funcionamiento de la aplicación para editar los datos de un Protegido.
}
		\UCitem{Actor}{Protector}
		\UCitem{Propósito}{Editar los datos de un Protegido}
		\UCitem{Entradas}{
			Se podrá modificar la información:
			\begin{itemize}
				\item Nombre Completo.
				\item Correo Electrónico.
				\item Numero de Celular.
				\item Contraseña.
			\end{itemize}
		 }
		\UCitem{Salidas}{
			\begin{itemize}
				\item \MSGref{MSJ1.1}{Campos Incompletos}
				\item \MSGref{MSJ1.2}{Formato de nombre incorrecto}
				\item \MSGref{MSJ1.3}{Formato de teléfono incorrecto}
				\item \MSGref{MSJ1.4}{Formato de contraseña incorrecto}
				\item \MSGref{MSJ1.5}{Contraseña no coincide}
				\item \MSGref{MSJ1.10}{Formato de correo incorrecto}
				\item \MSGref{MSJ1.6}{Error en conexión}
				\item \MSGref{MSJ1.9}{Contraseña incorrecta}
				\item \MSGref{MSJ1.8}{Datos ya utilizados}
				\item \MSGref{MSJ1.7}{Autentificación}
			\end{itemize}
		}
		\UCitem{Precondiciones}{
			El usuario Protector solo podrá modificar los datos de un Protegido.
		}
		\UCitem{Postcondiciones}{
			El usuario Protector podrá tener actualizada la información del usuario Protegido.
		}
		\UCitem{Autor}{
			\begin{itemize}
				\item Salas Hernández Abiran Natanael
			\end{itemize}
		}
	\end{UseCase}
	%-------------------------------------- COMIENZA descripción Trayectoria Crear
	\begin{UCtrayectoria}{Principal}
		\UCpaso Reproducir el \UCref{CU1.1}
		\UCpaso[\UCactor] Selecciona de la lista de contactos el Protegido a modificar.
		\UCpaso[\UCsist] Desplegara 3 opciones \IUbutton{Ubicar} \IUbutton{Editar} \IUbutton{Eliminar}
		\UCpaso[\UCactor] Selecciona el botón \IUbutton{Editar}.		
		\UCpaso[\UCactor] Despliega la pantalla \IUref{IU1.7}{Editar Protegido}
		\UCpaso[\UCactor] Selecciona el campo a modificar.
		\UCpaso[\UCactor] Pulsa en el  \IUbutton{GUARDAR}.
		\UCpaso[\UCsist] Verifica la regla de negocio \BRref{RN1.1}{Campos nulos}. \Trayref{A}
		\UCpaso[\UCsist] Verifica la regla de negocio \BRref{RN1.2}{Formato general de registro}.. \Trayref{B}
		\UCpaso[\UCsist] Despliega la pantalla \IUref{IU1.5}{Seguridad}.
		\UCpaso[\UCactor] Captura su contraseña en la pantalla.
		\UCpaso[\UCactor] Pulsa en el  \IUbutton{CONTINUAR}.
		\UCpaso[\UCsist] Entra a la base de datos con los datos ingresados. \Trayref{C} \Trayref{D}
		\UCpaso[\UCsist] Verifica la regla de negocio \BRref{RN1.3}{Usuario registrado}. \Trayref{E}
		\UCpaso[\UCsist] Despliega la pantalla \IUref{IU1.3}{Pantalla Inicial} con el usuario Protegido editado.
	\end{UCtrayectoria}
	%-------------------------------------Trayectoras alternativas
	\begin{UCtrayectoriaA}{A}{Condición: Faltan datos en el formulario.}
		\UCpaso[\UCsist] Muestra mensaje \MSGref{MSJ1.1}{Campos Incompletos}
		\UCpaso[\UCsist] Continúa trayectoria en el paso anterior de la trayectoria  principal que invocó esta trayectoria alternativa. \Trayref{Principal}
	\end{UCtrayectoriaA}
	%-------------------------------------Trayectoras alternativas
	\begin{UCtrayectoriaA}{B}{Condición: El actor captura la información de manera incorrecta.}
		\UCpaso[\UCsist] Dependiendo del error sera el mensaje:
			\begin{itemize}
			\item \MSGref{MSJ1.2}{Formato de nombre incorrecto}
			\item \MSGref{MSJ1.3}{Formato de teléfono incorrecto}
			\item \MSGref{MSJ1.4}{Formato de contraseña incorrecto}
			\item \MSGref{MSJ1.5}{Contraseña no coincide}
			\item \MSGref{MSJ1.10}{Formato de correo incorrecto}
			\end{itemize}
		\UCpaso[\UCsist] Continúa trayectoria en el paso anterior de la trayectoria principal que invocó esta trayectoria alternativa. \Trayref{Principal}
	\end{UCtrayectoriaA}
	%-------------------------------------Trayectoras alternativas
	\begin{UCtrayectoriaA}{C}{Condición: Falló la conexión en la base de datos}
		\UCpaso[\UCsist] El sistema mostrará el mensaje \MSGref{MSJ1.6}{Error en conexión}
		\UCpaso[\UCsist] Regresa al paso 1 de la trayectoria principal. \Trayref{Principal}
	\end{UCtrayectoriaA}
	%-------------------------------------Trayectoras alternativas
	\begin{UCtrayectoriaA}{D}{Condición: Contraseña incorrecta}
		\UCpaso[\UCsist] El sistema mostrará el mensaje \MSGref{MSJ1.9}{Contraseña incorrecta}
		\UCpaso[\UCsist] Continúa trayectoria en el paso anterior de la trayectoria  principal que invocó esta trayectoria alternativa. \Trayref{Principal}
	\end{UCtrayectoriaA}
	%-------------------------------------Trayectoras alternativas
	\begin{UCtrayectoriaA}{E}{Condición: Usuario ya registrado y/o correo y/o teléfono ya utilizados}
		\UCpaso[\UCsist] El sistema mostrará el mensaje \MSGref{MSJ1.8}{Datos ya utilizados}
		\UCpaso[\UCsist] Continúa trayectoria en el paso 4 de la trayectoria  principal que invocó esta trayectoria alternativa. \Trayref{Principal}
	\end{UCtrayectoriaA}
