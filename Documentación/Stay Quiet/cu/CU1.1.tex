\begin{UseCase}{CU1.1}{Iniciar Sesión}{
	Se describe el comportamiento y funcionamiento de la aplicación para el inicio de sesión mediante la autentificación.
}
		\UCitem{Versión}{1.0}
		\UCitem{Actor}{Protector, Protegido.}
		\UCitem{Propósito}{Tener un control del ingreso de los diferentes usuarios que hacen uso de la aplicación y así poder hacer uso de sus funciones permitidas.}
		\UCitem{Entradas}{
			Se ingresa la información:
			\begin{itemize}
				\item Correo Electrónico.
				\item Contraseña y/o PIN.
			\end{itemize}
		 }
		\UCitem{Salidas}{
			\begin{itemize}
				\item \MSGref{MSJ1.1}{Campos Incompletos}
				\item \MSGref{MSJ1.6}{Error en conexión}
				\item \MSGref{MSJ1.7}{Autentificación}
				\item \MSGref{MSJ1.10}{Formato de correo incorrecto}
			\end{itemize}
		}
		\UCitem{Precondiciones}{El usuario debe estar registrado en el sistema.
		}
		\UCitem{Postcondiciones}{
			El usuario podrá hacer uso de la aplicación según sea su perfil.
		}
		\UCitem{Autor}{
			\begin{itemize}
				\item Salas Hernández Abiran Natanael
			\end{itemize}
		}
		\UCitem{Estatus}{Revisión}
	\end{UseCase}
	%-------------------------------------- COMIENZA descripción Trayectoria Crear
	\begin{UCtrayectoria}{Principal}
		\UCpaso[\UCactor] Pulsa en el icono de la aplicación.
		\UCpaso[\UCsist] Despliega la pantalla \IUref{IU1.1}{Pantalla Principal}.
		\UCpaso[\UCactor] Captura los datos necesarios en la misma pantalla.
		\UCpaso[\UCactor] Pulsa en el checkbox para recordar los datos.
		\UCpaso[\UCactor] Pulsa en el botón \IUbutton{ENTRAR} de la pantalla.
		\UCpaso[\UCsist] Verifica la regla de negocio \BRref{RN1.1}{Campos nulos}. \Trayref{A}
		\UCpaso[\UCsist] Verifica la regla de negocio \BRref{RN1.2}{Formato general para el registro de un Protector}  para el formato del correo. \Trayref{B}
		\UCpaso[\UCsist] Verifica la regla de negocio (Checkbox). \Trayref{C}
		\UCpaso[\UCsist] Entra a la base de datos con los datos ingresados. \Trayref{D}		
		\UCpaso[\UCsist] Verifica la regla de negocio \BRref{RN1.3}{Usuario registrado en el sistema}. \Trayref{E}
		\UCpaso[\UCsist] Muestra la pantalla \IUref{IU1.3}{Pantalla Inicial}.
	\end{UCtrayectoria}
	%-------------------------------------Trayectoras alternativas
	\begin{UCtrayectoriaA}{A}{Condición: Faltan datos en el formulario.}
		\UCpaso[\UCsist] Muestra mensaje \MSGref{MSJ1.1}{Campos Incompletos}
		\UCpaso[\UCsist] Continúa trayectoria en el paso anterior de la trayectoria principal que invocó esta trayectoria alternativa. \Trayref{Principal}
	\end{UCtrayectoriaA}
	%-------------------------------------Trayectoras alternativas
	\begin{UCtrayectoriaA}{B}{Condición: El actor ingresa formato de correo incorrecto.}
		\UCpaso[\UCsist] Muestra mensaje \MSGref{MSJ1.10}{Formato de correo incorrecto}
		\UCpaso[\UCsist] Continúa trayectoria en el paso anterior de la trayectoria principal que invocó esta trayectoria alternativa. \Trayref{Principal}
	\end{UCtrayectoriaA}
	%-------------------------------------Trayectoras alternativas
	\begin{UCtrayectoriaA}{C}{Condición: No se pulso el checkbox}
		\UCpaso[\UCsist] El sistema no guardara el usuario y contraseña. 
		\UCpaso[\UCsist] Continúa trayectoria en el paso siguiente de la trayectoria principal que invocó esta trayectoria alternativa. \Trayref{Principal}
	\end{UCtrayectoriaA}
	%-------------------------------------Trayectoras alternativas
	\begin{UCtrayectoriaA}{D}{Condición: Falló la conexión en la base de datos}
		\UCpaso[\UCsist] El sistema mostrará el mensaje \MSGref{MSJ1.6}{Error en conexión}
		\UCpaso[\UCsist] Regresa al paso 1 de la trayectoria principal. \Trayref{Principal}
	\end{UCtrayectoriaA}
	%-------------------------------------Trayectoras alternativas
		\begin{UCtrayectoriaA}{E}{Condición: Usuario no registrado o correo y/o contraseña incorrectos}
			\UCpaso[\UCsist] El sistema mostrará el mensaje \MSGref{MSJ1.7}{Autentificación}
			\UCpaso[\UCsist] Continúa trayectoria en el paso anterior de la trayectoria principal que invocó esta trayectoria alternativa. \Trayref{Principal}
	\end{UCtrayectoriaA}
	%-------------------------------------Trayectoras alternativas
